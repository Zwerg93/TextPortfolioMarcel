\begin{figure}[h][p]
    \centering
    \includegraphics[scale=0.8]{pics/Screenshot from 2023-02-06 12-26-42.png}
    \caption{Eröterung: Definition + Aufbau}
    \label{fig:impl:eröterung1}
\end{figure}

\begin{figure}[h][p]
    \centering
    \includegraphics[scale=0.8]{pics/Screenshot from 2023-02-06 12-27-06.png}
    \caption{Eröterung: Verfassen}
    \label{fig:impl:eröterung2}
\end{figure}
\begin{figure}[h][p]
    \centering
    \includegraphics[scale=0.8]{pics/Screenshot from 2023-02-06 12-27-29.png}
    \caption{Eröterung: Fortsetzung}
    \label{fig:impl:eröterung3}
\end{figure}

\section{Mustertext}
Eine Schule  eine Einheit?  

Weltweit gibt es unzählige Schulen, die auf das Tragen einer Schuluniform bestehen. In Österreich findet man dies jedoch nur selten. Nun überlegt die HTL Leonding, ob mit kommendem Schuljahr die Uniformpflicht eingeführt werden soll. Diese Frage verlangt eine Klärung der wichtigsten Pro- und Kontraargumente, um eine passende Entscheidung zu gewährleisten. 

Der wohl wichtigste Pluspunkt für SchülerInnen ist die enorme Zeitersparnis am Morgen. Kein lästiges Kleidungsuchen und unzähliges Umziehen mehr. Man schlüpft einfach in die Uniform und schon ist man bereit für den Schultag. Die eingesparten Minuten können für ein ausgiebiges Frühstück oder einen Powernap benutzt werden.  

Für Eltern zeigt sich ein gewaltiger Vorteil in den finanziellen Unkosten. Schulkleidung kann sehr oft im Klassenverband bestellt werden, wodurch die Preise sinken. Die Ersparnis im Vergleich zu den neuesten Fashiontrends und Modemarken kann hier gravierend sein. Dies könnte vor allem finanziell benachteiligten Familien zugutekommen.  

Ein nicht zu verachtender Vorteil ist außerdem das Gemeinschaftsgefühl, das durch Uniformen ausgelöst wird. Die SchülerInnen einer Schule können sich mit dem Schullogo und den Farben identifizieren und stehen sowohl für die Ausbildungsstätte als auch füreinander ein. Damit kann die Lernatmosphäre ebenso positiv beeinflusst werden. 

Jedoch zeigen sich bei genauerer Betrachtung nicht nur positive Aspekte. Um ein stimmiges Bild der Thematik zeichnen zu können, soll nun auch den negativen Argumenten Platz eingeräumt werden.  

Negativ anzumerken ist, dass durch eine Schuluniform das Merkmal eines Individuums, die persönliche Note, die sich oft im Aussehen (Kleidung, Frisur) widerspiegelt, vollkommen verschwinden würde. Zur Folge hätte dies, dass alle gleich aussehen und niemand mehr seinen eigenen Stil ausleben könnte.  

Weiters zu bedenken ist, dass die Uniform eventuell mit Textilien produziert wird, welche nicht den persönlichen Bedürfnissen entspricht. Viele SchülerInnen haben vielleicht schon ihren Lieblingsstoff, in dem sie sich wohl fühlen, gefunden. Wenn die Schuluniform einem anderen Material entspricht, wäre dieses Wohlgefühl nicht mehr gegeben.  

Ein wesentlicher Punkt, der gegen eine einheitliche Schulkleidung spricht, ist die Stärkung gängiger Geschlechterrollen und -stereotypen. Sieht eine Uniform die klassischen Klischees von Hosen bei Jungs und Röcken bei Mädchen vor, so werden genderdiverse Personen außer Acht gelassen. Dies wäre eine weitere Benachteiligung dieser Gesellschaftsgruppe. 

Schlussfolgernd lässt sich feststellen, dass eine verpflichtende Schuluniform zwar den finanziellen und gesellschaftlichen Unterschied mancher SchülerInnen ausgleichen könnte, der Preis dafür jedoch zu hoch ist. Die Individualität und Persönlichkeit jedes Einzelnen/jeder Einzelnen muss in unserer diversen Gesellschaft höhere Priorität haben. 

390 Wörter 

\newpage
\section{Eigener Text}
\subsection{Angabe}
\subsubsection{Angabe}
\subsubsection{Thema 3: Entscheidungen treffen }
Verfassen Sie eine Erörterung.
Lesen Sie die Auszüge aus dem Bericht Saubere Mode hat’s schwer (Textbeilage 1) von der
Website der Umwelt-Organisation Greenpeace vom März 2015 und betrachten Sie die mit dem
Bericht veröffentlichte Informationsgrafik Auswahlkriterien für Mode/Kleidung (Textbeilage 2).
Verfassen Sie nun die Erörterung und bearbeiten Sie dabei die folgenden Arbeitsaufträge:

\begin{compactitem}
    \item Fassen Sie zentrale Informationen des Berichts (Textbeilage 1) kurz zusammen
    \item Analysieren Sie anhand der Informationsgrafik (Textbeilage 2), nach welchen Kriterien
    Jugendliche Kleidung auswählen. 
    \item Diskutieren Sie den derzeitigen Kleiderkonsum Jugendlicher, insbesondere die Diskrepanz
    zwischen Wissen und Kaufentscheidung. Berücksichtigen Sie dabei Auswirkungen auf Umwelt und Produktionsbedingungen.
    \item Machen Sie Vorschläge, wie jugendliche Konsumentinnen und Konsumenten mit den gewonnenen Erkenntnissen umgehen sollten.
\end{compactitem}


\subsubsection{Saubere Mode hat's schwer}
Kleidung wird immer billiger –
und immer mehr zur EinwegWare. Ein T-Shirt für 2,99 Euro?
Keine Seltenheit. Eine Shorts für
3,49  Euro? Die Regel. Fast im
Wochentempo eröffnen Billigketten neue Filialen in deutschen
Städten. Die Online-Angebote
von Firmen und digitalen Marktplätzen wie Amazon erobern
rasant Marktanteile. Im Schnitt
kauft jeder Deutsche fünf neue
Kleidungsstücke pro Monat –
Jugendliche eher mehr. Damit
hat sich der Konsum von Kleidung vom Jahr 2000 bis 2010 fast
verdoppelt.
Dieser überbordende Kleiderkonsum mag für uns bezahlbar sein –
der Planet dagegen kann ihn sich
nicht mehr leisten. In den asiatischen Produktionsländern vergiftet die rasant wachsende Textilindustrie die Trinkwasserressourcen.
Allein in China sind 320  Millionen Menschen ohne Zugang
zu sauberem Trinkwasser. Über
60 Prozent der Trinkwasserreserven der großen Städte Chinas sind
ernsthaft verschmutzt. Viele der in
der Textilproduktion eingesetzten
Chemikalien sind krebserregend,
hormonell wirksam oder toxisch
für Wasserorganismen. Und sie
finden sich inzwischen überall –
in der Küstenluft vor Südafrika, in
der Leber von Eisbären und in der
Muttermilch. Greenpeace kämpft
seit Jahren mit der Detox-Kampagne […] für eine saubere Textilindustrie. Doch um Wasser und
Gesundheit rund um den Globus
wirklich zu schützen, müssen wir
unseren Kleiderkonsum verändern. Greenpeace hat daher die
Konsumenten von morgen – die
Teenager – nach ihrem Einkaufsverhalten gefragt. Wir wollten
wissen: Welche Kleidung kauft
die Jugend heute und warum?
Wo informieren sich Jugendliche
über Mode, wo kaufen sie, wer
bezahlt? Muss das Teil vor allem
neu und billig sein, vor allem
schick – oder zählt das Leben der
Fabrikarbeiterinnen in Bangladesch auch etwas? Die hier vorliegende repräsentative Umfrage
(durchgeführt von Nuggets –
Market Research  \& Consulting
GmbH) unter 502  Jugendlichen
im Alter von 12 bis 19 Jahren in
Deutschland zeichnet ein umfassendes Bild des Kleiderkonsums
der Konsumenten von morgen.
Wissensstand zur Textilproduktion
Jugendliche sind informiert über
soziale und ökologische Missstände in der Textilproduktion
und wünschen sich mehr praktische Informationen und Einkaufshilfen.
Jugendliche wissen, dass die Textilproduktion Probleme verursacht.
Zum Beispiel ist 83  Prozent
bewusst, dass Kleidung mit
gefährlichen Chemikalien bearbeitet wird. Nahezu jeder (96 Prozent) hat zumindest davon gehört,
dass Arbeiter in der Modeindustrie zum Teil schlecht behandelt werden. Und sie wollen
mehr Information: Jeder zweite
Jugendliche würde gerne mehr
darüber wissen, wie die Kleidung der Lieblingsmarken hergestellt wird. Fast genauso viele
Jugendliche geben an, dass ihnen
der Zugang zu diesen Informationen fehlt. Nur 3–6 Prozent der
Jugendlichen kennen bekannte
Öko-Marken wie Armed Angels
oder Nudie Jeans. Auch wo man
fair oder bio produzierte Kleidung
bekommt, wissen sie oft nicht.
Informationsquellen und
Auswahlkriterien
Grün denken, konventionell kaufen: Design und Preis bestimmen
den Kauf
Die Jugendlichen sammeln
Ideen und Informationen über
Modetrends vor allem im privaten Umfeld (58 Prozent) und im
Netz: 43 Prozent der Jugendlichen
informieren sich auf Shoppingseiten wie Amazon oder Zalando
über Mode und 35 Prozent direkt
über das Webangebot von Marken. Bei den 18- bis 19-Jährigen ist
der Einfluss von Shoppingseiten

sogar schon bedeutender als
Informationen von Freunden und
Bekannten. Auch über andere,
Unabhängigkeit suggerierende
Informationsquellen wie ModeBlogs bewerben Unternehmen
ihre Produkte. Damit kommen
immer mehr Informationen aus
direkter Hand der Hersteller oder
von digitalen Marktplätzen mit
einem unmittelbaren Umsatzinteresse.


\subsection{Text}
\subsubsection{Mode: Eine Sünde?}
Wer hat nicht schon vor der Entscheidung gestanden, ob man das Teure oder das billige, knallige T-Shirt einpacken soll. Meist greift man dann zu dem, welches im Trend liegt und am besten noch so günstig wie möglich. Wer möchte denn schon viel Geld ausgeben, wenn man auch für weniger Geld großartige Sachen bekommen kann? Doch die wenigsten beschäftigen sich mit den Folgen solcher Entscheidungen. Die wenigsten wissen, wie schlecht diese billigen Modestücke wirklich sind. Um genau das geht es in dem Artikel “Saubere Mode hats schwer”, welcher im März 2015 auf der Seite von Greenpeace veröffentlicht wurde. 

 

In dem Artikel wurde vor allem das Konsumverhalten der jugendlichen Menschen untersucht. Und genau wird es wirklich heikel: Denn junge Menschen sind besonders anfällig auf die sogenannte “Fast Fashion”. Denn jeder, der häufig soziale Medien wie Instagram und Facebook benutzt, kommt mit den neusten Modetrends in Berührung. Und da die Jugendlichen sich sehr schnell ein Vorbild suchen, dem sie nacheifern können, ist Mode ein großes Geschäft geworden. Und die Auswirkungen sind fatal, denn genau diese Menschen wollen dann aussehen wie ihre Vorbilder, und kaufen dann blind bei Marken wie “Shein” ein. Und das führt zu einer Kettenreaktion, dass noch mehr billig Mode unter schlechten Bedingungen hergestellt wird. 

 

Das dieser Konsum große Auswirkungen auf die Umwelt hat, ist klar. Doch was kaum jemand dabei beachtet, ist die Herstellung dieser Kleidung. Ein T-Shirt für 4 Euro? Nehm ich, was kann denn schon schiefgehen. Dass aber für dieses Shirt wahrscheinlich jemand sehr hart für sehr wenig Geld arbeiten musste, ist einem dann egal. Oder man denkt einfach nicht daran, woher das Shirt kommt. Grüne Labels können ein guter Anfang sein, doch meistens ist die Kleidung nicht ausreichend markiert. Und mal ehrlich, es gibt so viele Öko Labels, da blickt doch keiner mehr durch. Hier müsste man was grundliegendes in unserer Gesellschaft ändern, um den Menschen klarzumachen, dass hinter jedem Produkt ein Mensch steht, der genau jenes herstellen musste. 

Die jungen Menschen müssen sich ihrer Handlungen bewusst werden und anfangen, sich zu informieren und anhand dieses Wissens Entscheidungen zu treffen. Und damit stehen wir wieder am Anfang vor zwei Kleidungsartikel. Einem günstigen und einem teuren Artikel. Und dann entscheiden Sie sich für denjenigen, welcher weniger Auswirkung auf die Umwelt und die Arbeitsbedingungen anderer Länder hat. \newpage
\section{Formulierungshilfen}
Einleitung:

\begin{compactitem}

    \item Das Thema XY wird seit einiger Zeit in den Medien viel diskutiert 
    \item  In der Öffentlichkeit wird in letzter Zeit kontrovers diskutiert, ob... 
    \item  In unserer Schule ist seit geraumer Zeit XY ein Problem und wird daher rege diskutiert. 
    \item Welche Argumente dafür bzw. dagegen sprechen, möchte ich im Folgenden erörtern. 
    \item Im nachfolgenden Text werde ich mich mit diesem Thema auseinandersetzen. 
    \item Inwieweit dieser Forderung entsprochen bzw. widersprochen werden kann, soll Thema dieser Erörterung sein. 
\end{compactitem}
Hauptteil: 
\begin{compactitem}
    \item Argumente der Gegenposition 
    \begin{compactitem}
        \item Ein immer wieder vorgebrachtes Argument der Gegenposition ist... 
        \item Um zu einem ausgewogenen Urteil zu kommen, werde ich zunächst die Argumente der Gegenseite erläutern. Das stärkste Argument lautet hier...
        \item Ein starkes Argument dafür/dagegen ist, dass... 
        \item Was zunächst dafür/dagegen spricht, ist... 
        \item Ein weiteres Argument für/gegen XY ist, dass... 
    \end{compactitem}
    \item Überleitung/Drehpunkt 
    \begin{compactitem}
        \item Die angeführten Argumente zeigen recht deutlich, dass ..., trotzdem gibt es gute  
        \item Gründe gegen/für den Vorschlag. 
        \item Trotz dieser Gründe gegen/für xy sind viele dennoch der Meinung, dass...  
        \item Hierfür haben sie auch gute Gründe. Zum einen führen sie an, dass … 
    \end{compactitem}
    \item Argumente der eigenen Position 
    \begin{compactitem}
        \item Was zunächst dafür/dagegen spricht, ist... 
        \item Ein weiteres Argument für/gegen XY ist, dass... 
        \item Ganz besonders betonen möchte ich, dass...
        \item  Es darf außerdem nicht vergessen werden, dass... 
        \item Von zentraler Bedeutung ist, dass...
        \item Mein stärkstes Argument ist, dass... 
    \end{compactitem}
    \item Beispiel/Beleg 
    \begin{compactitem}
        \item Ein Beispiel/Beleg dafür ist...
        \item Dies lässt sich durch folgendes Beispiel verdeutlichen/veranschaulichen: ... 
        \item Zu diesem Argument kann folgendes/r Beispiel/Beleg angeführt werden: … 
    \end{compactitem}
\end{compactitem}
Schluss: 
\begin{compactitem}
    \item Aus den zuletzt genannten Gründen vertrete ich den Standpunkt, dass... 
    \item Nach einer ausführlichen Abwägung der Vor- und Nachteile komme ich zu dem Schluss, dass... 
    \item Als Kompromiss schlage ich vor, dass ...  
    \item Eine mögliche Lösung wäre … 
\end{compactitem}

\subsection{Realitätsbezug}

Die Erörterung kommt in vielen Bereichen des echten Lebens vor.  Bereiche dafür wären Politik, Diskussionen, Debattenreden, Zeitschriftartikel und vieles mehr. Dabei ist es wichtig, um Argumente und Schlussfolgerungen zu bewerten und zu beurteilen, ob sie tatsächlich zutreffend und überzeugend sind. In einer Erörterung sollten Fakten und Daten verwendet werden, die auf verlässlichen Quellen basieren, um die Realität so genau wie möglich wiederzugeben. Zudem sollten verschiedene Perspektiven berücksichtigt werden, um ein breites Verständnis der Thematik zu erlangen. 
\subsection{Erklärung}
Eine Erörterung ist ein argumentativer Text, in dem ein bestimmtes Thema aus verschiedenen Perspektiven beleuchtet und bewertet wird. In einer Erörterung werden Argumente und Gegenargumente präsentiert, um eine bestimmte These zu unterstützen oder zu widerlegen.

Eine Erörterung hat in der Regel eine klare Einleitung, in der das Thema vorgestellt und die zu erörternde Frage formuliert wird. Darauf folgt der Hauptteil, in dem die Argumente für und gegen die These präsentiert werden. Im Schlussteil werden die Argumente zusammengefasst und eine begründete Meinung oder Empfehlung gegeben.

Sie ollte gut strukturiet und überzeugend sein. Es ist wichtig, dass die Argumente gut gewählt und begründet werden und dass sowohl die Argumente für als auch die Argumente gegen die These ausgebaut und dem Leser präsentiert werden. Zudem soll eine Erörterung objektiv und sachlich bleiben.

\subsection{Beispiele für verwandte Textsorten}
Empfehlung, Kommentar, Leserbrief, offener Brief

\subsection{Abgrenzung}
Die Besonderheit und damit allgemeine Abgrenzung der Erörterung
liegt darin, dass sie eine schulische Textsorte ist, d. h. außerhalb des
schulischen Kontexts nicht ohne weitere Konkretisierung (Anlass, Format etc.) vorkommt. Das bedeutet konsequenterweise auch, dass in
den verwandten Textsorten durchaus erörterungsähnliche Muster der
Argumentation auftreten können aber nicht müssen.

\subsection{Umfang}
405 bis 495 oder 540 bis 660 Wörter

\subsection{situativer Kontext}
kein von der Prüfungssituation abweichender Kontext
\subsection{Eigene Erfahrung} 
Hier ist es für mich schwerer, die Argumente genau abzugrenzen, und den Aufbau einzuhalten. Aber mit ein bisschen Übung ist auch das bei der Matura kein Problem