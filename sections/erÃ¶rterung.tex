\input{./sections/erörterungImg.tex}

\section{Mustertext}
Eine Schule  eine Einheit?  

Weltweit gibt es unzählige Schulen, die auf das Tragen einer Schuluniform bestehen. In Österreich findet man dies jedoch nur selten. Nun überlegt die HTL Leonding, ob mit kommendem Schuljahr die Uniformpflicht eingeführt werden soll. Diese Frage verlangt eine Klärung der wichtigsten Pro- und Kontraargumente, um eine passende Entscheidung zu gewährleisten. 

Der wohl wichtigste Pluspunkt für SchülerInnen ist die enorme Zeitersparnis am Morgen. Kein lästiges Kleidungsuchen und unzähliges Umziehen mehr. Man schlüpft einfach in die Uniform und schon ist man bereit für den Schultag. Die eingesparten Minuten können für ein ausgiebiges Frühstück oder einen Powernap benutzt werden.  

Für Eltern zeigt sich ein gewaltiger Vorteil in den finanziellen Unkosten. Schulkleidung kann sehr oft im Klassenverband bestellt werden, wodurch die Preise sinken. Die Ersparnis im Vergleich zu den neuesten Fashiontrends und Modemarken kann hier gravierend sein. Dies könnte vor allem finanziell benachteiligten Familien zugutekommen.  

Ein nicht zu verachtender Vorteil ist außerdem das Gemeinschaftsgefühl, das durch Uniformen ausgelöst wird. Die SchülerInnen einer Schule können sich mit dem Schullogo und den Farben identifizieren und stehen sowohl für die Ausbildungsstätte als auch füreinander ein. Damit kann die Lernatmosphäre ebenso positiv beeinflusst werden. 

Jedoch zeigen sich bei genauerer Betrachtung nicht nur positive Aspekte. Um ein stimmiges Bild der Thematik zeichnen zu können, soll nun auch den negativen Argumenten Platz eingeräumt werden.  

Negativ anzumerken ist, dass durch eine Schuluniform das Merkmal eines Individuums, die persönliche Note, die sich oft im Aussehen (Kleidung, Frisur) widerspiegelt, vollkommen verschwinden würde. Zur Folge hätte dies, dass alle gleich aussehen und niemand mehr seinen eigenen Stil ausleben könnte.  

Weiters zu bedenken ist, dass die Uniform eventuell mit Textilien produziert wird, welche nicht den persönlichen Bedürfnissen entspricht. Viele SchülerInnen haben vielleicht schon ihren Lieblingsstoff, in dem sie sich wohl fühlen, gefunden. Wenn die Schuluniform einem anderen Material entspricht, wäre dieses Wohlgefühl nicht mehr gegeben.  

Ein wesentlicher Punkt, der gegen eine einheitliche Schulkleidung spricht, ist die Stärkung gängiger Geschlechterrollen und -stereotypen. Sieht eine Uniform die klassischen Klischees von Hosen bei Jungs und Röcken bei Mädchen vor, so werden genderdiverse Personen außer Acht gelassen. Dies wäre eine weitere Benachteiligung dieser Gesellschaftsgruppe. 

Schlussfolgernd lässt sich feststellen, dass eine verpflichtende Schuluniform zwar den finanziellen und gesellschaftlichen Unterschied mancher SchülerInnen ausgleichen könnte, der Preis dafür jedoch zu hoch ist. Die Individualität und Persönlichkeit jedes Einzelnen/jeder Einzelnen muss in unserer diversen Gesellschaft höhere Priorität haben. 

390 Wörter 

\section{Eigener Text}
\subsubsection{Angabe}
\subsubsection{Thema 3: Entscheidungen treffen }
Verfassen Sie eine Erörterung.
Lesen Sie die Auszüge aus dem Bericht Saubere Mode hat’s schwer (Textbeilage 1) von der
Website der Umwelt-Organisation Greenpeace vom März 2015 und betrachten Sie die mit dem
Bericht veröffentlichte Informationsgrafik Auswahlkriterien für Mode/Kleidung (Textbeilage 2).
Verfassen Sie nun die Erörterung und bearbeiten Sie dabei die folgenden Arbeitsaufträge:

\begin{compactitem}
    \item Fassen Sie zentrale Informationen des Berichts (Textbeilage 1) kurz zusammen
    \item Analysieren Sie anhand der Informationsgrafik (Textbeilage 2), nach welchen Kriterien
    Jugendliche Kleidung auswählen. 
    \item Diskutieren Sie den derzeitigen Kleiderkonsum Jugendlicher, insbesondere die Diskrepanz
    zwischen Wissen und Kaufentscheidung. Berücksichtigen Sie dabei Auswirkungen auf Umwelt und Produktionsbedingungen.
    \item Machen Sie Vorschläge, wie jugendliche Konsumentinnen und Konsumenten mit den gewonnenen Erkenntnissen umgehen sollten.
\end{compactitem}

\subsubsection{Mode: Eine Sünde?}
Wer hat nicht schon vor der Entscheidung gestanden, ob man das Teure oder das billige, knallige T-Shirt einpacken soll. Meist greift man dann zu dem, welches im Trend liegt und am besten noch so günstig wie möglich. Wer möchte denn schon viel Geld ausgeben, wenn man auch für weniger Geld großartige Sachen bekommen kann? Doch die wenigsten beschäftigen sich mit den Folgen solcher Entscheidungen. Die wenigsten wissen, wie schlecht diese billigen Modestücke wirklich sind. Um genau das geht es in dem Artikel “Saubere Mode hats schwer”, welcher im März 2015 auf der Seite von Greenpeace veröffentlicht wurde. 

 

In dem Artikel geht es um die Herstellung von Billig-Mode, das Konsumverhalten der Menschen und die Auswirkungen. Vor allem Jugendliche wurden hierbei genau unter die Lupe genommen, da diese am häufigsten von solchen Kleidungsstücken profitieren. Im Durchschnitt kauft jeder Deutsche fünf neue Kleidungsstücke im Monat, Jugendliche deutlich mehr. Außerdem beschäftigt sich der Artikel mit den Informationsquellen der Jugendlichen und zeigt auf, dass solche oft nicht ausreichend oder lückenhaft sind. Viele von den jungen Menschen Wissen zwar, dass vor allem billige Kleidung unter schlechten Bedingungen hergestellt werden und auch oft schlecht für die Umwelt sind. Die wichtigsten Kriterien der Jugendlichen sind dabei vor allem das Aussehen der Kleidung, der Preis, die Qualität und die Marke. Siegel und das Herstellungsland der Kleidung ist dabei selten der Entscheidungsgrund, und genau das sollte sich ändern.  

 

Man müsste den Jugendlichen die Möglichkeit geben, sich zu informieren. Viele von ihnen wissen zwar, dass günstige Kleidung nicht gut für die Umwelt ist, doch welche Schäden die Herstellung wirklich hinterlässt, kennt kaum jemand. Viele werden nur von dem günstigen Preis, aktuellen Trends und große Vorbilder geblendet. Sie wollen nur das neuste Aktuellste. Das, was jeder trägt. Doch kaum jemand will die Folgen dieses Konsums wahrhaben. Die Ausbeutung, die Umweltverschmutzung und vieles mehr. Ich selbst schaue auch oftmals nicht genau auf die Herstellung, vor allem wenn ich mit meiner Familie einkaufen gehe. Das Problem dabei ist nicht, dass es mich nicht interessieren würde. Viel mehr gibt es einfach keine guten Quellen, wo man sich unabhängig informieren kann. Klar, manche Modefirmen wie Shein haben sogenannte “grüne Label”, doch diese sind alle undurchsichtig. Viele von diesen eigenen Labels sind auch nachweislich einfach nur leere Versprechen. Doch als Konsument sieht man das leider nie, einzig der niedrige Preis wird an den Verbraucher übergeben.  

Ein anderes Problem ist der Konsum an sich. Klar, hier könnte man jetzt den Konsum im generellen Kritisieren, aber hier soll es einfach über das Bedürfnis gehen, immer die neuste und tollste Mode besitzen zu müssen. Hier haben vor allem Social Media und das Fernsehen dazu beigetragen, Mode für jeden interessant zu machen. Dadurch möchten sich vor allem junge Menschen, die sich leichter beeinflussen lassen, immer auf den neusten Modetrend aufspringen. Vor allem hier müsste man sehr viel Aufwand in die Aufklärung stecken, um diesen Unsinn zu ändern.  

Die jungen Menschen müssen sich ihrer Handlungen bewusst werden und Anfangen, sich zu informieren und anhand dieses Wissens Entscheidungen zu treffen. Und damit stehen wir wieder am Anfang vor zwei Kleidungsartikel. Einem günstigen und einem teuren Artikel. Und dann entscheiden Sie sich für denjenigen, welcher weniger Auswirkung auf die Umwelt und die Arbeitsbedingungen anderer Länder hat. 

\section{Formulierungshilfen}
Einleitung:

\begin{compactitem}

    \item Das Thema XY wird seit einiger Zeit in den Medien viel diskutiert 
    \item  In der Öffentlichkeit wird in letzter Zeit kontrovers diskutiert, ob... 
    \item  In unserer Schule ist seit geraumer Zeit XY ein Problem und wird daher rege diskutiert. 
    \item Welche Argumente dafür bzw. dagegen sprechen, möchte ich im Folgenden erörtern. 
    \item Im nachfolgenden Text werde ich mich mit diesem Thema auseinandersetzen. 
    \item Inwieweit dieser Forderung entsprochen bzw. widersprochen werden kann, soll Thema dieser Erörterung sein. 
\end{compactitem}
Hauptteil: 
\begin{compactitem}
    \item Argumente der Gegenposition 
    \begin{compactitem}
        \item Ein immer wieder vorgebrachtes Argument der Gegenposition ist... 
        \item Um zu einem ausgewogenen Urteil zu kommen, werde ich zunächst die Argumente der Gegenseite erläutern. Das stärkste Argument lautet hier...
        \item Ein starkes Argument dafür/dagegen ist, dass... 
        \item Was zunächst dafür/dagegen spricht, ist... 
        \item Ein weiteres Argument für/gegen XY ist, dass... 
    \end{compactitem}
    \item Überleitung/Drehpunkt 
    \begin{compactitem}
        \item Die angeführten Argumente zeigen recht deutlich, dass ..., trotzdem gibt es gute  
        \item Gründe gegen/für den Vorschlag. 
        \item Trotz dieser Gründe gegen/für xy sind viele dennoch der Meinung, dass...  
        \item Hierfür haben sie auch gute Gründe. Zum einen führen sie an, dass … 
    \end{compactitem}
    \item Argumente der eigenen Position 
    \begin{compactitem}
        \item Was zunächst dafür/dagegen spricht, ist... 
        \item Ein weiteres Argument für/gegen XY ist, dass... 
        \item Ganz besonders betonen möchte ich, dass...
        \item  Es darf außerdem nicht vergessen werden, dass... 
        \item Von zentraler Bedeutung ist, dass...
        \item Mein stärkstes Argument ist, dass... 
    \end{compactitem}
    \item Beispiel/Beleg 
    \begin{compactitem}
        \item Ein Beispiel/Beleg dafür ist...
        \item Dies lässt sich durch folgendes Beispiel verdeutlichen/veranschaulichen: ... 
        \item Zu diesem Argument kann folgendes/r Beispiel/Beleg angeführt werden: … 
    \end{compactitem}
\end{compactitem}
Schluss: 
\begin{compactitem}
    \item Aus den zuletzt genannten Gründen vertrete ich den Standpunkt, dass... 
    \item Nach einer ausführlichen Abwägung der Vor- und Nachteile komme ich zu dem Schluss, dass... 
    \item Als Kompromiss schlage ich vor, dass ...  
    \item Eine mögliche Lösung wäre … 
\end{compactitem}

\subsection{Realitätsbezug}

Die Erörterung kommt in vielen Bereichen des echten Lebens vor.  Bereiche dafür wären Politik, Diskussionen, Debattenreden, Zeitschriftartikel und vieles mehr. Dabei ist es wichtig, um Argumente und Schlussfolgerungen zu bewerten und zu beurteilen, ob sie tatsächlich zutreffend und überzeugend sind. In einer Erörterung sollten Fakten und Daten verwendet werden, die auf verlässlichen Quellen basieren, um die Realität so genau wie möglich wiederzugeben. Zudem sollten verschiedene Perspektiven berücksichtigt werden, um ein breites Verständnis der Thematik zu erlangen. 