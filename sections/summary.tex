\section{Fehlersammlung}

Übergänge und vor allem Rechtschreibung fällt mir schwer. Teilweise werde ich einfach zu unkonzentriert, und lese mir die Texte nicht mehr genau durch, und übersehe dadurch schnell kleine Fehler, die sich schnell zusammenläppern. Bei den Übergängen bin ich mir bei den Formulierungen oft nicht sicher, wie ich am besten eine flüssige Überleitung schreiben soll. Ich werde mir vor der Matura mehr von den Formulierungshilfen merken, um die Übergänge besser zu gestalten, und mir mehr Zeit zum Durchlesen nehmen. Inhaltlich gibt es kaum bis keine Probleme, aufpassen muss ich jedoch, dass der Inhalt zu den Aufgabenstellungen passt, und dass ich nicht einfach so drauf los schreibe. Dadurch kann der Text auch schnell mal zu lange werden, und mehr Fehler passieren. Da ich das aber niemals vorhersagen kann, werde ich einfach mehr üben müssen. Vor allem, weil ich oft in einen Flow reinkomme, wo mir das Schreiben guter Texte einfach fällt. Oft, aber nicht immer. Und dann sitzt ich manchmal da, und weiß nicht, was ich schreiben soll. 

Probleme bei dem Portfolio gab es hauptsächlich bei den neuen Texten, und vor allem auch zeitlich. Es war einfach sehr viel zu tun, wir haben alle viel zu spät angefangen. Selber schuld, werden Sie sich jetzt denken, aber es ging einfach nicht früher. Es war viel zu viel zu tun, und ich bin mir sicher, ich war nicht der Einzige, der kurz vor dem Zusammenbruch stand, weil er einfach viel zu überlastet war. Es wäre einfach nicht möglich gewesen, so viel früher anzufangen und somit mehr auf Fehler und Inhalt zu achten. Ich bin froh, dass ich so weit gekommen bin, eigentlich alles abgeschlossen habe. Ich bin nicht zufrieden, aber es ist da.  

\section{Schlusswort}
Viele der Textsorten waren keine große Herausforderung mehr, da wir sie die letzten Jahre oft und gut genug geübt haben. Einzig die Zusammenfassung und die Erörterung waren Probleme, da beide schon Jahre her sind. Verbessert habe ich mich vor allem bei den Formulierungen und meinem Ausdruck, da ich natürlich jetzt viel mehr Erfahrung als beim ersten Versuch habe.  

\section{Danksagung}
Am Ende dieser langen Reise möchte ich mich bei meinen liebsten bedanken. Bei Chen, meine Freundin, die mich immer motiviert hat, weiterzumachen. Sie war immer da, und vor allem in den schweren Zeiten hat es mir sehr geholfen, aufzustehen, und weiterzuschreiben. Außerdem möchte ich Lorenz danken, Lorenz aka die Lanze. Er war immer eine zuverlässige Quelle zu Informationen, und hat mir bei meiner Recherche sehr geholfen. Eine weitere wichtige Person auf dieser Reise war David Ursprung (auch als DJ XuX bekannt), der an dunklen Tagen immer den richtigen Musikmix bereit hatte. Philipp war ein Geist, der immer hinter einem stand, selbst wenn man ihn nicht sah. Sami hatte immer einen Überblick über alles, und half uns, alles zu koordinieren. Altenhofer David sorgte immer für die benötigte Ruhe im Klassenzimmer, während uns Mathias mit seinen Witzen immer unterhalten hatte. Dave war wie eine Mentale stütze, und wusste immer, jeden zu motivieren. Immer an seiner Seite war Ema, die nicht nur die Mutter der Klasse war, sondern jeden wieder zurück ins Leben gerufen hatte. Fabian Maar war der Analytische, der Ruhige. Er unterstütze jeden, wo er nur konnte, und schrieb still und heimlich an seinem Portfolio. Ana sorgte für die willkommene Abwechslung, mit lustigen Sprüchen und frechen Kommentaren. Annika konnte die Klasse mit dem besten Essen versorgen, während wir alle Stundenlang über unseren Laptops hingen.  Fabian Ettinger war der Professionelle, der immer genug Energie (drinks) für die ganze Klasse hatte. Im Speedrun Style hat er das Portfolio innerhalb weniger Tagen aus seinem Kopf gezaubert. Dominik war der Träumer, er träumte von einer Welt, wo nur die ganze Klasse zusammen ein Portfolio schreiben muss. Doch er stand immer hinter einem, und hat jeden unterstütz.  

 
\section{Quellen}

Erörterung Aufgabenstellungen: \href{https://prod.aufgabenpool.at/srdp\_vgs/aufgaben/60/KL18\_PT1\_DEU\_3.1\_AU.pdf}{https://prod.aufgabenpool.at/srdp\_vgs/aufgaben/60/KL18\_PT1\_DEU\_3.1\_AU.pdf}  

Kommentar Aufgabenstellungen:  \href{https://www.deutsche-grammatik.net/textsorten-srdp/kommentar/ }{https://www.deutsche-grammatik.net/textsorten-srdp/kommentar/ }

\href{https://prod.aufgabenpool.at/srdp\_vgs/aufgaben/267/KL22\_PT1\_DEU\_2.2\_AU.pdf}{https://prod.aufgabenpool.at/srdp\_vgs/aufgaben/267/KL22\_PT1\_DEU\_2.2\_AU.pdf}

Zusammenfassung Beispieltext: \href{https://www.scribbr.de/studium/zusammenfassung-schreiben/}{https://www.scribbr.de/studium/zusammenfassung-schreiben/}

Zusammenfassung aufgabenstellung \href{https://www.deutsche-grammatik.net/app/download/19417237025/Zusammenfassung\_Die+junge+Generation+ist+benachteiligt.pdf?t=1666362281 }{https://www.deutsche-grammatik.net/app/download/19417237025/Zusammenfassung\_Die+junge+Generation+ist+benachteiligt.pdf?t=1666362281 }

Kommentar: \href{ https://www.deutsche-grammatik.net/textsorten-srdp/kommentar/}{ https://www.deutsche-grammatik.net/textsorten-srdp/kommentar/}

Aufgabenpool: \href{https://prod.aufgabenpool.at/srdp/startseite/}{https://prod.aufgabenpool.at/srdp/startseite/}

Informationen: \href{https://www.matura.gv.at/index.php?eID=dumpFile\&t=f\&f=4525\&token=950c7f2b86f0ebc3459c5f0aa0e04013ab99c572}{https://www.matura.gv.at/index.php?eID=dumpFile\&t=f\&f=4525\&token=950c7f2b86f0ebc3459c5f0aa0e04013ab99c572}