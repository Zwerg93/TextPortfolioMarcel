\thispagestyle{empty}
\vspace{3cm}
~ \\ \\

Vorwort: 

 

Also, dass hier ist also das berüchtigte Vorwort. ‚Warum?‘ Werdet ihr euch sicher fragen? Warum hat er genau diesen Einstieg gewählt? Ganz einfach. Weil ich mir schwer mit Anfängen tue. Ich muss erstmal was haben, um dann in einen Flow zu kommen, wo die Text dann wie aus einem gebrochenen Damm aus meinen Fingern fließen. Grundsätzlich, wenn es um mein Wissen um die Textsorten geht, fühl ich mich schon ziemlich gut vorbereitet. Wieso? Weil ich gerne schreibe. Egal, ob irgendwelche Kurztexte, bei denen ich die Namen der Personen ändern sollte (tut mir übrigens leid, dass ich Sie eingebaut habe), oder irgendwelche Kommentare, Texte oder anderes. Sogar eine Weiterführung einer Buchreihe wollte ich schreiben, habe schon angefangen, mir die Handlung auszudenken, die Personen zu charakterisieren, und Nebenschauplätze zu entwerfen. Wieso dieses Buch noch nicht in den Regalen von Thalia steht? Weil ich leider bei solchen Projekten nie die Motivation habe, sie fertig zu machen.  

Da ich wirklich viel und gerne lese, habe ich über die Jahre einen guten Eindruck bekommen, wie man eine Geschichte Strukturiert, wie man künstlich Spannung aufbaut, und viele kleine, aber wichtige Details im laufe der Geschichte wieder zur Sprache bringt. Wieso ich dieses Talent nicht nutze? Faulheit. Ich habe meistens nicht die Motivation, mich aufzuraffen und neues zu schaffen. Auch bei der Rechtschreibung habe ich so meine Probleme, vor allem, weil ich mir meine Texte zu selten und ungenau durchlese. Das wird vor allem bei der Diplomarbeit ein Problem werden, dass ist aber noch ein ganz anderes Thema. Ich verlasse mich da einfach viel zu viel auf die technischen Möglichkeiten, und glaube, dass es damit getan ist. Aber im Grunde halte ich mich für einen recht guten Schreiberling, welcher mit etwas Mühe und der richtigen Motivation viel umsetzen könnte. Im laufe der Arbeit wird wahrscheinlich auch genau das mein Problem werden. Weiterzumachen. Mich aufzuraffen. Jeden Tag. Und mich an dieses Dokument zu setzten. Dabei nicht das Leben und meine Liebsten vergessen, sondern mit jedem die Zeit verbringen, die er verdient hat. 

% Hier kommt die Unterschrift drüber
\begin{tabbing}
Leonding, Februar 2022 \hspace{5cm} M. Pouget
\end{tabbing}
\vspace{10cm}
\newpage
\setcounter{page}{1}
